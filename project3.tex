\documentclass[]{article}

%packages
\usepackage[top=1in, bottom=1in, left=1in, right=1in]{geometry} %adjust margins. TODO: hack to fix large top margin
\usepackage{setspace} %allows doublespacing, onehalfspacing, singlespacing
\usepackage{enumitem} %for continuing lists
\usepackage{titling} %for moving the title
\usepackage[normalem]{ulem} %for underlining
\usepackage{graphicx} %for inserting images
\usepackage{listings} %for source code
\usepackage{color} %for defining colors (syntax highlighting)

\renewcommand{\thesubsection}{\thesection.\alph{subsection}}

%syntax highlighting
\definecolor{mygreen}{rgb}{0,0.6,0}
\definecolor{mygray}{rgb}{0.5,0.5,0.5}
\definecolor{mymauve}{rgb}{0.58,0,0.82}

\lstset{ %
  backgroundcolor=\color{white},   % choose the background color
  basicstyle=\footnotesize,        % size of fonts used for the code
  breaklines=true,                 % automatic line breaking only at whitespace
  captionpos=b,                    % sets the caption-position to bottom
  commentstyle=\color{mygreen},    % comment style
  escapeinside={\%*}{*)},          % if you want to add LaTeX within your code
  keywordstyle=\color{blue},       % keyword style
  stringstyle=\color{mymauve},     % string literal style
}

\begin{document}
\lstset{language=C++}

\begin{spacing}{.4}
\setlength{\droptitle}{-7em}
\title{CSE 430 Project 3}
\author{David Ganey}
\maketitle
\end{spacing}

\section{Introduction}
This project simulates a file system with basic creation, writing, and persistence features. The project demonstrates knowledge of Unix-style file systems and uses a variety of data structures and programming techniques. While the majority of the program is complete, some features are not fully operational (see section~\ref{sec:problems} for more information).

\section{Data Structures and Utility Functions}
This section will discuss some of the data structures and basic algorithms implemented for use in the File/Directory APIs.

\subsection{Bitmaps}
The bitmaps are implemented as arrays of characters to save space. Each character stores 1 byte, which can represent the state of 8 inodes or directories. Two methods were written to work with these bitmaps, called \texttt{int findFirstAvailableInode()} and \texttt{int findFirstAvailableDataSector}. Each of these functions loops through all characters in the bitmap, and for each one, checks each bit (from highest to lowest). When it finds a 0, the function flips it and returns the corresponding position. When the inode function returns, it returns the specific number of the inode. The caller then needs to deduce (using integer division) in which sector that inode is actually stored on the disk (since each sector contains four inodes). In contrast, the data sector function simply returns the actual sector to be used.

\subsection{Searching}
When creating files or directories, it is essential to traverse the file structure until the \emph{parent} of the new node is found. A recursive function, called \texttt{int searchInodeForPath}, accomplishes this. The parameters given to this function are \texttt{int inodeToSearch}, which starts as 0 (the root inode number), a vector of \texttt{std::strings} which represents the path (passed by reference to avoid copies), and \texttt{int pathSegment}, which tells the function which index of the vector to check. At each call of this function, it first checks if \texttt{pathSegment} is at the end. If the current segment is the end of the path (the name of the new directory/file), then \texttt{inodeToSearch} already represents the parent, and we simply return that. Otherwise, the function loads the current directory inode and searches the directory contents for names which match the vector entry at the current segment. When it finds one, it makes the recursive call:
\lstinputlisting[language=c++]{inodeSearch.cc}

\subsection{Arithmetic}
Basic integer division and modulo functions are heavily used in this project. Most sectors contain more than one entry (e.g. 4 inodes per sector), so some arithmetic is required to determine how to reference the specific section of the block and the specific sector in which the block should be written. The code below demonstrates this arithmetic, showing how an inode number of 6 could be used to write an inode to index 2 of sector 1:
\lstinputlisting[language=c++]{inodeArithmetic.cc}

\subsection{Open File Table}
The open file table is basically implemented as an array. However, to make removal of entries easier a map is used instead. The map maps from integers to a custom struct called \texttt{OpenFile}. This structure contains the inode number of the file and that file's current filepointer. When a file is opened, a new entry is added, and when a file is closed, that entry is removed.

\subsection{Other Structs}
Each type of sector which could be stored has a corresponding structure which, through casting, produces a program-usable version of the bytes stored on the disk. The most basic of these structs stores the data in a file data block, and simply contains a \texttt{char[SECTOR\_SIZE]}. More complicated structures, like the inode, contain multiple variables. For each structure which does not fully fill the sector, a character array called "garbage" was added to fill the remaining space. While C++ casting should simply truncate the end, this was helpful in ensuring that all structures are precisely the same size.

\section{API Implementation}
Most of the API functions share a similar structure. All functions with a \texttt{char* path} parameter start out by tokenizing the string, storing each ``/" delimited substring as a string in a vector. Any function which needs to add a directory or inode uses the bitmap functions to flip the appropriate bit and to determine which sector to store the new file in. This means that the bitmap is filled in order -- the first character starts as 00000000, then 10000000, then 11000000, etc. In a more complex filesystem which supports deleting files, a bitmap would be more useful, but for this project a bitmap could be substituted with a simple integer which keeps count of how many inodes are allocated. This would simply be incremented each time, instead of requiring more complicated bit manipulation.

\section{Problems} \label{sec:problems}
Testing this application was very difficult. While I believe I have thoroughly stepped through most functions in the Visual Studio debugger, even that powerful tool is not sufficient to fully verify the functionality of this file system. To truly ascertain whether all features of this application work, functions would be needed to read and print the contents of directories and files. Those could be used in conjunction with a battery of automated testing to ensure that files and directories are working properly.
\\ \\
This application performs best the first time it is started with a particular disk image. I believe some of the features do not work as reliably when the disk has resumed by loading a previous disk image. I have not yet been able to determine why this is.

\end{document}